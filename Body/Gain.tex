%!TEX root = ../BUSystematics.tex

\graphicspath{{Body/Figures/Gain/IFG/60h/Amplitude/}{Body/Figures/Gain/IFG/60h/Amplitude-With-AdHoc/}{Body/Figures/Gain/IFG/60h/Lifetime/}{Body/Figures/Gain/IFG/9d/Lifetime/}{Body/Figures/Gain/ResidualGain/EnergyBinKloss/}{Body/Figures/Gain/ResidualGain/Chi2Min/}}

\section{Gain Systematic Uncertainties}


Detected positon energies are corrected for in-fill, short-term, and out-of-fill gain corrections (IFG, STDP, OOF) \cite[and references therein]{GainNote}. A laser system was used to determine these corrections and the parameters associated with them. Uncertainties in the IFG and STDP correction parameters propagate into uncertainties on the extracted \wa values\footnote{The OOF correction acts on time scales much larger than a fill, and so does not add a systematic uncertainty.}. Lastly and separately, a systematic uncertainty was evaluated in regards to the possible need for a residual or ad-hoc gain correction.



\subsection{IFG Amplitude and Time Constant}


The IFG correction is given by
\begin{align}
  E_{t} = E_{m}/(1 - A_{IFG} e^{-t/\tau_{IFG}}),
\end{align}
where $E_{t}$ is the true energy of the detected positron, $E_{m}$ is it's measured energy, and $A_{IFG}$ and $\tau_{IFG}$ are the amplitude and time constant parameters used in the correction. The systematic uncertainties on \wa (or \R\footnote{\R is related to \wa via the equation
\begin{align}
  \omega_{a} = 2\pi \cdot \SI{0.2291}{MHz} \cdot (1 + (R - \Delta R) \times 10^{-6}),
\label{eq:wa}
\end{align}
where $\Delta R$ is a blinding offset.}) due to the uncertainties in the amplitude and time constant parameters were evaluated by scanning over those parameters in multiplicative steps, determining the senstivity of \R to those parameters, and multiplying by the respective parameter errors:
    \begin{align}
        \delta R = \sigma_{(A_{IFG}, \tau_{IFG})} \times \frac{dR}{d(A_{IFG}, \tau_{IFG})}.
    \end{align}


The uncertainties in the IFG (and STDP) parameters are given in \tabref{tab:gainCorrErrs}. These uncertainties were evaluated by D. Sweigart, where he calculated the average crystal parameter uncertainties weighted by the counts put into his T-Method histogram. Note that the IFG parameter uncertainties for the HighKick and Endgame datasets is noticeably less than those of the 60h and 9d datasets. This is because the time constants in the IFG corrections were set to those values determined from long-term double pulse studies (LTDP) \cite{GainNote}, rather than IFG correction studies\footnote{Connected to this fact is that for the 60h and 9d datasets, the gain corrections were applied in the order \{OOF, STDP, IFG\}, whereas for the HighKick and Endgame datasets, the order was \{STDP, OOF, IFG\}.}. This resulted in more accurately determined values and therefore smaller parameter uncertainties.


\begin{table}
\centering
\setlength\tabcolsep{10pt}
\renewcommand{\arraystretch}{1.2}
\begin{tabular*}{1\linewidth}{@{\extracolsep{\fill}}lHHHH}
  \hline
    \multicolumn{5}{c}{\textbf{Gain Correction Parameter Uncertainties}} \\
  \hline
    Quantity & \thead{60h} & \thead{HighKick} & \thead{9d} & \thead{Endgame} \\
  \hline
    IFG Amplitude      & 21.4 & 6.3 & 15.3  & 3.7 \\
    IFG Time Constant  & 16.1 & 6.5 & 11.5 & 6.5 \\
    STDP Amplitude     & 1.9 & 1.9 & 1.9 & 1.9 \\
    STDP Time Constant & 3.4  & 3.4 & 3.4 & 3.4 \\
  \hline 
\end{tabular*}
\caption[]{Average uncertainties on the crystal gain correction parameters, in perecent, weighted by the number of counts put into the T-Method histogram, as determined by D. Sweigart. For sources on the numbers see the left side of Table~6.5 \cite{phdthesis:2020Sweigart} for the 60h and 9d datasets, and the Run~1 uncertainty spreadsheet for the HK and EG datasets \cite{UncertaintySpreadsheet}. Original gain correction data can be found in \cite{GainElog1,GainElog2,GainElog3,GainElog4}. All STDP parameters were the same as they were calculated from the same laser data. The author calculated the average (not hit-weighted) uncertainties, and found less than a percent difference in each case.}
\label{tab:gainCorrErrs}
\end{table}




\figref{fig:IFGAmpscan} shows the fitted \R and \chisq values for the T- and R-Methods versus the multiplier on the IFG amplitude for the 60h dataset. The sensitivity is then determined by fitting a straight line to the plot in the range 0--2. These sensitivities in general reach a minimia when the fit start time is chosen such that it lies near a \gmtwo zero crossing. Similarly, Figures~\ref{fig:IFGtauscan} and \ref{fig:IFGtauscan9d} show the fitted \R and \chisq value versus the multiplier on the IFG time constant for the 60h and 9d datasets respectively. Whereas the trends for \R versus the amplitude are linear, the shape is a little more complex for the time constant. For the 60h and HighKick datasets the shape starts off flat at a multiplier of 0, dips down a bit as it approaches a multiplier of 1, and then rises linearly. For the 9d and Endgame datasets, the trend is the same, except that at higher multipliers around 1.5, the trend starts to drop. 




This meant that the lifetime parameters behaved slightly differently, as evidenced by the plots. In order to evaluate the errors consistently among the different datasets, the sensitivities were extracted by fitting a straight line to the plots with the same range. That range was determined via inspection of the various plots, and set to 0.8--1.5. 



\begin{figure}[h]
\centering
    \begin{subfigure}[t]{0.45\textwidth}
        \centering
        \includegraphics[width=\textwidth]{IFG_Amplitude_Compare_R}
        % \caption{}
    \end{subfigure}% %you need this % here to add spacing between subfigures
    \hspace{1cm}
    \begin{subfigure}[t]{0.45\textwidth}
        \centering
        \includegraphics[width=\textwidth]{IFG_Amplitude_Compare_Chisq}
        % \caption{}
    \end{subfigure}
\caption[]{\R (left) and \chisq (right) versus the multiplier on the IFG amplitude parameter, for the T- and R-Methods. The sensitivities of \R to the parameter is determined by fitting a line to the points in the range 0--2, and the extracted values are included in the plot in units of ppm per unit amplitude. It should be noted that the fluctuations in the points are statistical in nature. As shown the sensitivity of the R-Method to the IFG amplitude is less than that of the T-Method. Data are from the 60h dataset.}
\label{fig:IFGAmpscan}
\end{figure}



\begin{figure}[h]
\centering
    \begin{subfigure}[t]{0.45\textwidth}
        \centering
        \includegraphics[width=\textwidth]{IFG_Lifetime_Compare_R}
        % \caption{}
    \end{subfigure}% %you need this % here to add spacing between subfigures
    \hspace{1cm}
    \begin{subfigure}[t]{0.45\textwidth}
        \centering
        \includegraphics[width=\textwidth]{IFG_Lifetime_Compare_Chisq}
        % \caption{}
    \end{subfigure}
\caption[]{\R (left) and \chisq (right) versus the multiplier on the IFG time constant parameter, for the T- and R-Methods, for the 60h dataset. The sensitivities of \R to the parameter is determined by fitting a line to the points in the range 0.8--1.5, and the extracted values are included in the plot in units of ppm per unit time constant. As shown the sensitivity of the R-Method to the IFG amplitude is less than that of the T-Method.}
\label{fig:IFGtauscan}
\end{figure}


\begin{figure}[h]
\centering
    \begin{subfigure}[t]{0.45\textwidth}
        \centering
        \includegraphics[width=\textwidth]{IFG_Lifetime_Compare_R_9d}
        % \caption{}
    \end{subfigure}% %you need this % here to add spacing between subfigures
    \hspace{1cm}
    \begin{subfigure}[t]{0.45\textwidth}
        \centering
        \includegraphics[width=\textwidth]{IFG_Lifetime_Compare_Chisq_9d}
        % \caption{}
    \end{subfigure}
\caption[]{\R (left) and \chisq (right) versus the multiplier on the IFG time constant parameter, for the T- and R-Methods, for the 9d dataset. The sensitivities of \R to the parameter is determined by fitting a line to the points in the range 0.8--1.5, and the extracted values are included in the plot in units of ppm per unit time constant. As shown the sensitivity of the R-Method to the IFG amplitude is less than that of the T-Method. \R versus the time constant for the 9d and Endgame datasets falls off at high multipliers, due to a different procedure for determining the time constants in the IFG correction, than the 60h and HighKick datasets.}
\label{fig:IFGtauscan9d}
\end{figure}





\tabref{tab:IFGResults} gives the sensitivities to the IFG amplitude and time constant parameters for the four Run~1 datasets, and the associated systematic errors after those sensitivities have been multiplied against the corresponding errors in \tabref{tab:gainCorrErrs}. As shown the errors are small, $\mathcal{O}(10)$ ppb. 




\begin{table}[h]
\centering
% \setlength\tabcolsep{10pt}
\renewcommand{\arraystretch}{1.2}
\begin{tabularx}{0.9\linewidth}{XcYKYK}
  \hline
    \multicolumn{6}{c}{\textbf{IFG Amplitude Scan Results}} \\
  \hline\hline
    Dataset & \thead{Fit Method} & \multicolumn{1}{c}{$dR/dA_{IFG}$} & \multicolumn{1}{c}{$\boldsymbol{\delta R_{A_{IFG}}}$} & \multicolumn{1}{c}{$dR/d\tau_{IFG}$} & \multicolumn{1}{c}{$\boldsymbol{\delta R_{\tau_{IFG}}}$} \\
  \hline
    \multirow{2}{*}{60h} & T & 36.4 & 7.8 & 125.6 & 20.2 \\
                         & R & 12.7 & 2.7 & 72.6 & 11.7 \\
  \hline
    \multirow{2}{*}{HighKick} & T & 52.1 & 3.3 & 150.4 & 9.8 \\
                              & R & 8.4 & 0.5 & 49.7 & 3.2 \\
  \hline
    \multirow{2}{*}{9d} & T & 29.6 & 4.5 & 46.9 & 5.4 \\
                        & R & 9.7 & 1.5 & 8.2 & 0.9 \\
  \hline
    \multirow{2}{*}{Endgame} & T & 64.0 & 2.4 & 204.5 & 13.3 \\
                             & R & 27.9 & 1.0 & 101.3 & 6.6 \\
  \hline
\end{tabularx}
\caption[]{Sensitivities of \R to IFG amplitude parameters for the four Run~1 datasets, in units of ppb per unit amplitude.}
\label{tab:IFGResults}
\end{table}






% \begin{table}[h]
% \centering
% % \setlength\tabcolsep{10pt}
% \renewcommand{\arraystretch}{1.2}
% \begin{tabularx}{0.9\linewidth}{XOOJ}
%   \hline
%     \multicolumn{4}{c}{\textbf{IFG Amplitude Scan Results}} \\
%   \hline\hline
%     Dataset & \thead{Fit Method} & \multicolumn{1}{c}{$dR/dA_{IFG}$} & \multicolumn{1}{c}{$\boldsymbol{\delta R}$} \\
%   \hline
%     \multirow{2}{*}{60h} & T & 36.4 & 7.8 \\
%                          & R & 12.7 & 2.7 \\
%   \hline
%     \multirow{2}{*}{HighKick} & T & 52.1 & 3.3 \\
%                               & R & 8.4 & 0.5 \\
%   \hline
%     \multirow{2}{*}{9d} & T & 29.6 & 4.5 \\
%                         & R & 9.7 & 1.5 \\
%   \hline
%     \multirow{2}{*}{Endgame} & T & 64.0 & 2.4 \\
%                              & R & 27.9 & 1.0 \\
%   \hline
% \end{tabularx}
% \caption[]{Sensitivities of \R to IFG amplitude parameters for the four Run~1 datasets, in units of ppb per unit amplitude.}
% \label{tab:IFGampScan}
% \end{table}




% \begin{table}[h]
% \centering
% % \setlength\tabcolsep{10pt}
% \renewcommand{\arraystretch}{1.2}
% \begin{tabularx}{0.9\linewidth}{XOOK}
%   \hline
%     \multicolumn{4}{c}{\textbf{IFG Time Constant Scan Results}} \\
%   \hline\hline
%     Dataset & \thead{Fit Method} & \multicolumn{1}{c}{$dR/d\tau_{IFG}$} & \multicolumn{1}{c}{$\boldsymbol{\delta R}$} \\
%   \hline
%     \multirow{2}{*}{60h} & T & 125.6 & 20.2 \\
%                          & R & 72.6 & 11.7 \\
%   \hline
%     \multirow{2}{*}{HighKick} & T & 150.4 & 9.8 \\
%                               & R & 49.7 & 3.2 \\
%   \hline
%     \multirow{2}{*}{9d} & T & 46.9 & 5.4 \\
%                         & R & 8.2 & 0.9 \\
%   \hline
%     \multirow{2}{*}{Endgame} & T & 204.5 & 13.3 \\
%                              & R & 101.3 & 6.6 \\
%   \hline
% \end{tabularx}
% \caption[]{Sensitivities of \R to IFG time constant parameters for the four Run~1 datasets, in units of ppb per unit amplitude.}
% \label{tab:IFGtauScan}
% \end{table}




\clearpage
\subsection{STDP}




- turned it on and off 


% - mention STDP as the results from with vs without the effect - reiterate that with my randomization scheme I don't have to worry about the statistics part as much, briefly mention that I don't have the ability to scan over the effect since I do things at the histogram level - do mention that I use Aaron's gain corrector module to make the trees...






\begin{table}[h]
\centering
% \setlength\tabcolsep{10pt}
\renewcommand{\arraystretch}{1.2}
\begin{tabularx}{0.9\linewidth}{XOOK}
  \hline
    \multicolumn{4}{c}{\textbf{STDP Results}} \\
  \hline\hline
    Dataset & \thead{Fit Method} & \multicolumn{1}{c}{$\Delta R_{\text{(w/ - w/o)}}$} & \multicolumn{1}{c}{$\boldsymbol{\delta R}$} \\
  \hline
    \multirow{2}{*}{60h} & T & 4.8 & 0.1 \\
                         & R & 2.8 & 0.1 \\
  \hline
    \multirow{2}{*}{HighKick} & T & 2.2 & 0.1 \\
                              & R & 2.2 & 0.1 \\
  \hline
    \multirow{2}{*}{9d} & T & 11.2 & 0.2 \\
                        & R & 14.0 & 0.3 \\
  \hline
    \multirow{2}{*}{Endgame} & T & 3.9 & 0.1 \\
                             & R & 5.0 & 0.1 \\
  \hline
\end{tabularx}
\caption[]{}
\label{tab:systematicError_STDP}
\end{table}








\subsection{Residual Gain Correction}

% applied to the cluster energies in order to 

% -potentially talk about both ways to get the value for the amplitdue, chi2 min and flatten k_loss - and show plots for this
% -give equation for residual gain and show which values I used, and then cite Aaron's work for the value I used
% did energy bin fits with T method, only main CBO parameters, applied various ad hoc gain amplitudes to try and flatten out the spectrum
% - decide what I want to do for the ad hoc gain scan and then explain those results, include relevant plots, include updated plots for the IFG scan showing consistency with 1 afteer I have done so


\begin{figure}[h]
\centering
    \begin{subfigure}[t]{0.45\textwidth}
        \centering
        \includegraphics[width=\textwidth]{TMethod_Chi2_Vs_AdHocAmplitude_Canv}
        \caption{T-Method \chisq versus pileup multiplier. The parabolic fit equation used was $y = p_{2}(x - p_{1})^{2} + p_{0}.$}
    \end{subfigure}% %you need this % here to add spacing between subfigures
    \hspace{1cm}
    \begin{subfigure}[t]{0.45\textwidth}
        \centering
        \includegraphics[width=\textwidth]{TMethod_R_Vs_AdHocAmplitude_Canv}
        \caption{T-Method $R$ versus pileup multiplier. The parameter $p_{1}$ gives the sensitivity of $R$ to the value of the pileup multiplier, with units in ppm.}
    \end{subfigure}

    \begin{subfigure}[t]{0.45\textwidth}
        \centering
        \includegraphics[width=\textwidth]{FullRatio_Chi2_Vs_AdHocAmplitude_Canv}
        \caption{R-Method \chisq versus pileup multiplier. The parabolic fit equation used was $y = p_{2}(x - p_{1})^{2} + p_{0}.$}
    \end{subfigure}% %you need this % here to add spacing between subfigures
    \hspace{1cm}
    \begin{subfigure}[t]{0.45\textwidth}
        \centering
        \includegraphics[width=\textwidth]{FullRatio_R_Vs_AdHocAmplitude_Canv}
        \caption{R-Method $R$ versus pileup multiplier. The parameter $p_{1}$ gives the sensitivity of $R$ to the value of the pileup multiplier, with units in ppm.}
    \end{subfigure}
\caption[]{Fix all captions. Data are from the 60h dataset.}
\label{fig:AdHocGainScan}
\end{figure}





\begin{landscape}
\begin{figure}[h]
\centering
    \begin{subfigure}[t]{0.4\textwidth}
        \centering
        \includegraphics[width=\textwidth]{TMethod_kappa_loss_Vs_EBin_Canv_HK-0}
        \caption{}
    \end{subfigure}% %you need this % here to add spacing between subfigures
    \hspace{1cm}
    \begin{subfigure}[t]{0.4\textwidth}
        \centering
        \includegraphics[width=\textwidth]{TMethod_kappa_loss_Vs_EBin_Canv_HK-5p1}
        \caption{}
    \end{subfigure}
    \hspace{1cm}
    \begin{subfigure}[t]{0.4\textwidth}
        \centering
        \includegraphics[width=\textwidth]{TMethod_kappa_loss_Vs_EBin_Canv_HK-7p5}
        \caption{}
    \end{subfigure}

    \begin{subfigure}[t]{0.4\textwidth}
        \centering
        \includegraphics[width=\textwidth]{TMethod_kappa_loss_Vs_EBin_Canv_EG-0}
        \caption{}
    \end{subfigure}% %you need this % here to add spacing between subfigures
    \hspace{1cm}
    \begin{subfigure}[t]{0.4\textwidth}
        \centering
        \includegraphics[width=\textwidth]{TMethod_kappa_loss_Vs_EBin_Canv_EG-11p2}
        \caption{}
    \end{subfigure}
    \hspace{1cm}
    \begin{subfigure}[t]{0.4\textwidth}
        \centering
        \includegraphics[width=\textwidth]{TMethod_kappa_loss_Vs_EBin_Canv_EG-6p0}
        \caption{}
    \end{subfigure}
\caption[]{}
\label{fig:EBinKloss}
\end{figure}
\end{landscape}





\begin{table}[h]
\centering
% \setlength\tabcolsep{10pt}
\renewcommand{\arraystretch}{1.2}
\begin{tabularx}{0.9\linewidth}{XHGHG}
  \hline
    \multicolumn{5}{c}{\textbf{Residual Gain Tests}} \\
  \hline\hline
            & \multicolumn{2}{c}{$\chi^{2}_{min}$} & \multicolumn{2}{c}{flat $\kappa_{loss}$} \\
  \cmidrule(lr){2-3}\cmidrule(lr){4-5}
    Dataset & \multicolumn{1}{c}{$A_{g} \times 10^{-4}$} & \multicolumn{1}{c}{p value} & \multicolumn{1}{c}{$A_{g} \times 10^{-4}$} & \multicolumn{1}{c}{p value} \\
  \hline
    60h & 8.5 & 0.81 & 8.5 & 0.81 \\
    HighKick & 5.1 & \sim 0 & 7.5 & 0.11 \\
    9d & 8.4 & 0.17 & 10.0 & 0.25 \\
    Endgame & 11.2 & \sim 0 & 6.0 & 0.04 \\
  \hline
\end{tabularx}
\caption[]{}
\label{tab:AdHocGainTests}
\end{table}


\begin{table}[h]
\centering
% \setlength\tabcolsep{10pt}
\renewcommand{\arraystretch}{1.2}
\begin{tabularx}{0.9\linewidth}{XOOJ}
  \hline
    \multicolumn{4}{c}{\textbf{Systematic Error Due to Residual Gain}} \\
  \hline\hline
    Dataset & \thead{Fit Method} & \multicolumn{1}{c}{$ \chi^{2}_{min} \quad \delta R$} & \multicolumn{1}{c}{flat $\kappa_{loss}$ $\boldsymbol{\delta R}$}  \\
  \hline
    \multirow{2}{*}{60h} & T & 28.7 & 28.7 \\
                         & R & 22.9 & 22.9 \\
  \hline
    \multirow{2}{*}{HighKick} & T & 1.0 & 11.8 \\
                              & R & 1.8 & 15.3 \\
  \hline
    \multirow{2}{*}{9d} & T & 18.1 & 24.4 \\
                        & R & 15.2 & 19.1 \\
  \hline
    \multirow{2}{*}{Endgame} & T & 29.9 & 18.8 \\
                             & R & 8.5 & 6.7 \\
  \hline
\end{tabularx}
\caption[]{}
\label{tab:AdHocGainErr}
\end{table}



