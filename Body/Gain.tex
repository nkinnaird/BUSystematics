%!TEX root = ../BUSystematics.tex

\graphicspath{{Body/Figures/Gain/IFG/60h/Amplitude/}{Body/Figures/Gain/IFG/60h/Amplitude-With-AdHoc/}{Body/Figures/Gain/IFG/60h/Lifetime/}{Body/Figures/Gain/IFG/9d/Lifetime/}}

\section{Gain Systematic Errors}



-Describe generalities of gain effects here






\begin{table}
\centering
\setlength\tabcolsep{10pt}
\renewcommand{\arraystretch}{1.2}
\begin{tabular*}{1\linewidth}{@{\extracolsep{\fill}}lHHHH}
  \hline
    \multicolumn{5}{c}{\textbf{Gain Correction Parameter Errors}} \\
  \hline
    Quantity & \thead{60h} & \thead{HighKick} & \thead{9d} & \thead{Endgame} \\
  \hline
    IFG Amplitude      & 21.4 & 17.2  & 15.3  & 11.0 \\
    IFG Time Constant  & 16.1 & 13.2 & 11.5 & 7.5 \\
    STDP Amplitude     & 1.9 & 1.9 & 1.9 & 1.9 \\
    STDP Time Constant & 3.4  & 3.4 & 3.4 & 3.4 \\
  \hline 
\end{tabular*}
\caption[]{Average uncertainties on the crystal gain correction parameters, weighted by the number of counts put into the T-Method histogram, as determined by D. Sweigart, left side of Table 6.5 \cite{phdthesis:2020Sweigart}. Original data can be found in \cite{GainElog1,GainElog2,GainElog3,GainElog4}. In general the uncertainties on the IFG parameters decreased with statistics, as more laser data was acquired with which to determine the constants. All STDP parameters were the same as calculated from the same STDP laser data. The author calculated the average (not hit-weighted) uncertainties, and found less than a percent difference in each case. Units are in percent.}
\label{tab:gainCorrErrs}
\end{table}




% Keep in mind that for HighKick and Endgame the crystal gain lifetime parameters were fixed to tau_ltdp, using the same study. 

% For HighKick and Endgame the corrections go as STDP, OOF, IFG. 
% For 60h and 9d the corrections go as OOF, STDP, IFG.

% HighKick had some steps in the CTAGs.





% - need to compare R vs T method results
% - need to reference the errors used for the amplitude and the lifetime - hit weighted numbers David came up with for some datasets, and standard errors for the other datasets with the fixed tau ltdp
% -record those errors into my table so people know exactly what they are
% - describe the reasoning for the range for the linear fit in the lifetime case - perhaps include figures from other datasets showing the behavior
% - point out that these two errors are highly correlated but I'm being very conservative and adding them in quadrature
% - mention STDP as the results from with vs without the effect - reiterate that with my randomization scheme I don't have to worry about the statistics part as much, briefly mention that I don't have the ability to scan over the effect since I do things at the histogram level - do mention that I use Aaron's gain corrector module to make the trees...
% - decide what I want to do for the ad hoc gain scan and then explain those results, include relevant plots, include updated plots for the IFG scan showing consistency with 1 afteer I have done so





\subsection{IFG Amplitude}


The systematic error due to the IFG amplitude was determined by 




\begin{figure}[h]
\centering
    \begin{subfigure}[t]{0.45\textwidth}
        \centering
        \includegraphics[width=\textwidth]{IFG_Amplitude_Compare_R}
        \caption{}
    \end{subfigure}% %you need this % here to add spacing between subfigures
    \hspace{1cm}
    \begin{subfigure}[t]{0.45\textwidth}
        \centering
        \includegraphics[width=\textwidth]{IFG_Amplitude_Compare_Chisq.png}
        \caption{}
    \end{subfigure}
\caption[Systematic error due to]{Systematic error due to. Data are from the 60h dataset.}
\label{fig:IFGAmpscan}
\end{figure}




\begin{table}
\centering
\renewcommand{\arraystretch}{1.2}
\begin{tabularx}{0.65\linewidth}{@{\extracolsep{\fill}}XYY}
  \hline
    \multicolumn{3}{c}{\textbf{Systematic Error due to}} \\
  \hline\hline
    Dataset & \thead{T-Method} & \thead{R-Method} \\
  \hline
    60h & 0.0 & 0.0 \\
    HighKick & 0.0 & 0.0 \\
    9d & 0.0 & 0.0 \\ 
    Endgame & 0.0 & 0.0 \\
  \hline
\end{tabularx}
\caption[Systematic error due to]{Systematic error due to. Units are in ppb.}
\label{tab:systematicError_}
\end{table}




\subsection{IFG Time Constant}


- range from 0.8 to 1.5, determined from inspection of the various datasets, and then made the same for all datasets.


\begin{figure}[h]
\centering
    \begin{subfigure}[t]{0.45\textwidth}
        \centering
        \includegraphics[width=\textwidth]{IFG_Lifetime_Compare_R}
        \caption{}
    \end{subfigure}% %you need this % here to add spacing between subfigures
    \hspace{1cm}
    \begin{subfigure}[t]{0.45\textwidth}
        \centering
        \includegraphics[width=\textwidth]{IFG_Lifetime_Compare_Chisq}
        \caption{}
    \end{subfigure}
\caption[Systematic error due to]{Systematic error due to. Data are from the 60h dataset.}
\label{fig:IFGAmpscan}
\end{figure}


\begin{figure}[h]
\centering
    \begin{subfigure}[t]{0.45\textwidth}
        \centering
        \includegraphics[width=\textwidth]{IFG_Lifetime_Compare_R_9d}
        \caption{}
    \end{subfigure}% %you need this % here to add spacing between subfigures
    \hspace{1cm}
    \begin{subfigure}[t]{0.45\textwidth}
        \centering
        \includegraphics[width=\textwidth]{IFG_Lifetime_Compare_Chisq_9d}
        \caption{}
    \end{subfigure}
\caption[Systematic error due to]{Systematic error due to. Data are from the 9d dataset.}
\label{fig:IFGAmpscan}
\end{figure}


- should I include the awkward looking 9d plot here?


\begin{table}
\centering
\renewcommand{\arraystretch}{1.2}
\begin{tabularx}{0.65\linewidth}{@{\extracolsep{\fill}}XYY}
  \hline
    \multicolumn{3}{c}{\textbf{Systematic Error due to}} \\
  \hline\hline
    Dataset & \thead{T-Method} & \thead{R-Method} \\
  \hline
    60h & 0.0 & 0.0 \\
    HighKick & 0.0 & 0.0 \\
    9d & 0.0 & 0.0 \\ 
    Endgame & 0.0 & 0.0 \\
  \hline
\end{tabularx}
\caption[Systematic error due to]{Systematic error due to. Units are in ppb.}
\label{tab:systematicError_}
\end{table}





\subsection{STDP}


- turned it on and off 



\begin{table}
\centering
\renewcommand{\arraystretch}{1.2}
\begin{tabularx}{0.65\linewidth}{@{\extracolsep{\fill}}XYY}
  \hline
    \multicolumn{3}{c}{\textbf{Systematic Error due to}} \\
  \hline\hline
    Dataset & \thead{T-Method} & \thead{R-Method} \\
  \hline
    60h & 0.0 & 0.0 \\
    HighKick & 0.0 & 0.0 \\
    9d & 0.0 & 0.0 \\ 
    Endgame & 0.0 & 0.0 \\
  \hline
\end{tabularx}
\caption[Systematic error due to]{Systematic error due to. Units are in ppb.}
\label{tab:systematicError_}
\end{table}







\subsection{Residual Gain Variation}

\begin{table}
\centering
\renewcommand{\arraystretch}{1.2}
\begin{tabularx}{0.65\linewidth}{@{\extracolsep{\fill}}XYY}
  \hline
    \multicolumn{3}{c}{\textbf{Systematic Error due to}} \\
  \hline\hline
    Dataset & \thead{T-Method} & \thead{R-Method} \\
  \hline
    60h & 0.0 & 0.0 \\
    HighKick & 0.0 & 0.0 \\
    9d & 0.0 & 0.0 \\ 
    Endgame & 0.0 & 0.0 \\
  \hline
\end{tabularx}
\caption[Systematic error due to]{Systematic error due to. Units are in ppb.}
\label{tab:systematicError_}
\end{table}
