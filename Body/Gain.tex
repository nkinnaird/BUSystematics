%!TEX root = ../BUSystematics.tex

\graphicspath{{Body/Figures/Gain/IFG/60h/Amplitude/}{Body/Figures/Gain/IFG/60h/Amplitude-With-AdHoc/}{Body/Figures/Gain/IFG/60h/Lifetime/}{Body/Figures/Gain/IFG/9d/Lifetime/}{Body/Figures/Gain/ResidualGain/EnergyBinKloss/}{Body/Figures/Gain/ResidualGain/Chi2Min/}}

\section{Gain Systematic Uncertainties}


The detected positon energies are corrected for in-fill, short-term, and out-of-fill gain corrections (IFG, STDP, OOF) \cite[and references therein]{GainNote}. Uncertainties in the IFG and STDP corrections will propagate into uncertainties on the extracted \wa values\footnote{The OOF correction acts on time scales much larger than a fill, and so does not add a systematic uncertainty.}. For the 60h and 9d datasets, the corrections went as \{OOF, STDP, IFG\}, whereas for the HighKick and Endgame datasets, the corrections went as \{STDP, OOF, IFG\}. A laser system was used to determine these corrections and the parameters associated with them. 




Lastly, a systematic uncertainty due to the application of a residual or ad-hoc gain correction which fixes some issues in the analysis is evaluated.






\subsection{IFG Amplitude and Time Constant}

The IFG correction is given by
\begin{align}
  E_{t} = E_{m}/(1 - A_{IFG} e^{-t/\tau_{IFG}}),
\end{align}
where $E_{t}$ is the true energy of the detected positron, $E_{m}$ is it's measured energy, and then $A_{IFG}$ and $\tau_{IFG}$ are the amplitude and time constant parameters used in the parameterization of the IFG gain correction. The systematic uncertainties due to the uncertainties in the amplitude and time constant are evaluated by scanning over those parameters in multiplicative steps, determining the senstivity of \R to the 





\begin{table}
\centering
\setlength\tabcolsep{10pt}
\renewcommand{\arraystretch}{1.2}
\begin{tabular*}{1\linewidth}{@{\extracolsep{\fill}}lHHHH}
  \hline
    \multicolumn{5}{c}{\textbf{Gain Correction Parameter Uncertainties}} \\
  \hline
    Quantity & \thead{60h} & \thead{HighKick} & \thead{9d} & \thead{Endgame} \\
  \hline
    IFG Amplitude      & 21.4 & 6.3 & 15.3  & 3.7 \\
    IFG Time Constant  & 16.1 & 6.5 & 11.5 & 6.5 \\
    STDP Amplitude     & 1.9 & 1.9 & 1.9 & 1.9 \\
    STDP Time Constant & 3.4  & 3.4 & 3.4 & 3.4 \\
  \hline 
\end{tabular*}
\caption[]{Average uncertainties on the crystal gain correction parameters, weighted by the number of counts put into the T-Method histogram, as determined by D. Sweigart, left side of Table 6.5 \cite{phdthesis:2020Sweigart} for 60h and 9d datasets, and personal communication/spreadsheet for the HK and EG datasets \cite{UncertaintySpreadsheet} (which used the tau ltdp). Original data can be found in \cite{GainElog1,GainElog2,GainElog3,GainElog4}. All STDP parameters were the same as calculated from the same STDP laser data. The author calculated the average (not hit-weighted) uncertainties, and found less than a percent difference in each case. Units are in percent.}
\label{tab:gainCorrErrs}
\end{table}







% Keep in mind that for HighKick and Endgame the crystal gain lifetime parameters were fixed to tau_ltdp, using the same study. 

% For HighKick and Endgame the corrections go as STDP, OOF, IFG. 
% For 60h and 9d the corrections go as OOF, STDP, IFG.








The systematic error due to the IFG amplitude was determined by 




\begin{figure}[h]
\centering
    \begin{subfigure}[t]{0.45\textwidth}
        \centering
        \includegraphics[width=\textwidth]{IFG_Amplitude_Compare_R}
        \caption{}
    \end{subfigure}% %you need this % here to add spacing between subfigures
    \hspace{1cm}
    \begin{subfigure}[t]{0.45\textwidth}
        \centering
        \includegraphics[width=\textwidth]{IFG_Amplitude_Compare_Chisq.png}
        \caption{}
    \end{subfigure}
\caption[Systematic error due to]{Systematic error due to. Data are from the 60h dataset.}
\label{fig:IFGAmpscan}
\end{figure}





\begin{table}[h]
\centering
% \setlength\tabcolsep{10pt}
\renewcommand{\arraystretch}{1.2}
\begin{tabularx}{0.9\linewidth}{XOOJ}
  \hline
    \multicolumn{4}{c}{\textbf{IFG Amplitude Scan Results}} \\
  \hline\hline
    Dataset & \thead{Fit Method} & \multicolumn{1}{c}{$dR/dA_{ifg}$} & \multicolumn{1}{c}{$\boldsymbol{\delta R}$} \\
  \hline
    \multirow{2}{*}{60h} & T & 36.4 & 7.8 \\
                         & R & 12.7 & 2.7 \\
  \hline
    \multirow{2}{*}{HighKick} & T & 52.1 & 3.3 \\
                              & R & 8.4 & 0.5 \\
  \hline
    \multirow{2}{*}{9d} & T & 29.6 & 4.5 \\
                        & R & 9.7 & 1.5 \\
  \hline
    \multirow{2}{*}{Endgame} & T & 64.0 & 2.4 \\
                             & R & 27.9 & 1.0 \\
  \hline
\end{tabularx}
\caption[]{}
\label{tab:IFGampScan}
\end{table}







- range from 0.8 to 1.5, determined from inspection of the various datasets, and then made the same for all datasets.


\begin{figure}[h]
\centering
    \begin{subfigure}[t]{0.45\textwidth}
        \centering
        \includegraphics[width=\textwidth]{IFG_Lifetime_Compare_R}
        \caption{}
    \end{subfigure}% %you need this % here to add spacing between subfigures
    \hspace{1cm}
    \begin{subfigure}[t]{0.45\textwidth}
        \centering
        \includegraphics[width=\textwidth]{IFG_Lifetime_Compare_Chisq}
        \caption{}
    \end{subfigure}
\caption[Systematic error due to]{Systematic error due to. Data are from the 60h dataset.}
\label{fig:IFGAmpscan}
\end{figure}


\begin{figure}[h]
\centering
    \begin{subfigure}[t]{0.45\textwidth}
        \centering
        \includegraphics[width=\textwidth]{IFG_Lifetime_Compare_R_9d}
        \caption{}
    \end{subfigure}% %you need this % here to add spacing between subfigures
    \hspace{1cm}
    \begin{subfigure}[t]{0.45\textwidth}
        \centering
        \includegraphics[width=\textwidth]{IFG_Lifetime_Compare_Chisq_9d}
        \caption{}
    \end{subfigure}
\caption[]{Systematic error due to. Data are from the 9d dataset.}
\label{fig:IFGAmpscan}
\end{figure}


- should I include the awkward looking 9d plot here?




\begin{table}[h]
\centering
% \setlength\tabcolsep{10pt}
\renewcommand{\arraystretch}{1.2}
\begin{tabularx}{0.9\linewidth}{XOOK}
  \hline
    \multicolumn{4}{c}{\textbf{IFG Time Constant Scan Results}} \\
  \hline\hline
    Dataset & \thead{Fit Method} & \multicolumn{1}{c}{$dR/d\tau_{ifg}$} & \multicolumn{1}{c}{$\boldsymbol{\delta R}$} \\
  \hline
    \multirow{2}{*}{60h} & T & 125.6 & 20.2 \\
                         & R & 72.6 & 11.7 \\
  \hline
    \multirow{2}{*}{HighKick} & T & 150.4 & 9.8 \\
                              & R & 49.7 & 3.2 \\
  \hline
    \multirow{2}{*}{9d} & T & 46.9 & 5.4 \\
                        & R & 8.2 & 0.9 \\
  \hline
    \multirow{2}{*}{Endgame} & T & 204.5 & 13.3 \\
                             & R & 101.3 & 6.6 \\
  \hline
\end{tabularx}
\caption[]{}
\label{tab:IFGtauScan}
\end{table}






\subsection{STDP}




- turned it on and off 


% - mention STDP as the results from with vs without the effect - reiterate that with my randomization scheme I don't have to worry about the statistics part as much, briefly mention that I don't have the ability to scan over the effect since I do things at the histogram level - do mention that I use Aaron's gain corrector module to make the trees...






\begin{table}[h]
\centering
% \setlength\tabcolsep{10pt}
\renewcommand{\arraystretch}{1.2}
\begin{tabularx}{0.9\linewidth}{XOOK}
  \hline
    \multicolumn{4}{c}{\textbf{STDP Results}} \\
  \hline\hline
    Dataset & \thead{Fit Method} & \multicolumn{1}{c}{$\Delta R_{\text{(w/ - w/o)}}$} & \multicolumn{1}{c}{$\boldsymbol{\delta R}$} \\
  \hline
    \multirow{2}{*}{60h} & T & 4.8 & 0.1 \\
                         & R & 2.8 & 0.1 \\
  \hline
    \multirow{2}{*}{HighKick} & T & 2.2 & 0.1 \\
                              & R & 2.2 & 0.1 \\
  \hline
    \multirow{2}{*}{9d} & T & 11.2 & 0.2 \\
                        & R & 14.0 & 0.3 \\
  \hline
    \multirow{2}{*}{Endgame} & T & 3.9 & 0.1 \\
                             & R & 5.0 & 0.1 \\
  \hline
\end{tabularx}
\caption[]{}
\label{tab:systematicError_STDP}
\end{table}








\subsection{Residual Gain Variation}

% applied to the cluster energies in order to 

% -potentially talk about both ways to get the value for the amplitdue, chi2 min and flatten k_loss - and show plots for this
% -give equation for residual gain and show which values I used, and then cite Aaron's work for the value I used
% did energy bin fits with T method, only main CBO parameters, applied various ad hoc gain amplitudes to try and flatten out the spectrum
% - decide what I want to do for the ad hoc gain scan and then explain those results, include relevant plots, include updated plots for the IFG scan showing consistency with 1 afteer I have done so


\begin{figure}[h]
\centering
    \begin{subfigure}[t]{0.45\textwidth}
        \centering
        \includegraphics[width=\textwidth]{TMethod_Chi2_Vs_AdHocAmplitude_Canv}
        \caption{T-Method \chisq versus pileup multiplier. The parabolic fit equation used was $y = p_{2}(x - p_{1})^{2} + p_{0}.$}
    \end{subfigure}% %you need this % here to add spacing between subfigures
    \hspace{1cm}
    \begin{subfigure}[t]{0.45\textwidth}
        \centering
        \includegraphics[width=\textwidth]{TMethod_R_Vs_AdHocAmplitude_Canv}
        \caption{T-Method $R$ versus pileup multiplier. The parameter $p_{1}$ gives the sensitivity of $R$ to the value of the pileup multiplier, with units in ppm.}
    \end{subfigure}

    \begin{subfigure}[t]{0.45\textwidth}
        \centering
        \includegraphics[width=\textwidth]{FullRatio_Chi2_Vs_AdHocAmplitude_Canv}
        \caption{R-Method \chisq versus pileup multiplier. The parabolic fit equation used was $y = p_{2}(x - p_{1})^{2} + p_{0}.$}
    \end{subfigure}% %you need this % here to add spacing between subfigures
    \hspace{1cm}
    \begin{subfigure}[t]{0.45\textwidth}
        \centering
        \includegraphics[width=\textwidth]{FullRatio_R_Vs_AdHocAmplitude_Canv}
        \caption{R-Method $R$ versus pileup multiplier. The parameter $p_{1}$ gives the sensitivity of $R$ to the value of the pileup multiplier, with units in ppm.}
    \end{subfigure}
\caption[]{Fix all captions. Data are from the 60h dataset.}
\label{fig:AdHocGainScan}
\end{figure}





\begin{landscape}
\begin{figure}[h]
\centering
    \begin{subfigure}[t]{0.4\textwidth}
        \centering
        \includegraphics[width=\textwidth]{TMethod_kappa_loss_Vs_EBin_Canv_HK-0}
        \caption{}
    \end{subfigure}% %you need this % here to add spacing between subfigures
    \hspace{1cm}
    \begin{subfigure}[t]{0.4\textwidth}
        \centering
        \includegraphics[width=\textwidth]{TMethod_kappa_loss_Vs_EBin_Canv_HK-5p1}
        \caption{}
    \end{subfigure}
    \hspace{1cm}
    \begin{subfigure}[t]{0.4\textwidth}
        \centering
        \includegraphics[width=\textwidth]{TMethod_kappa_loss_Vs_EBin_Canv_HK-7p5}
        \caption{}
    \end{subfigure}

    \begin{subfigure}[t]{0.4\textwidth}
        \centering
        \includegraphics[width=\textwidth]{TMethod_kappa_loss_Vs_EBin_Canv_EG-0}
        \caption{}
    \end{subfigure}% %you need this % here to add spacing between subfigures
    \hspace{1cm}
    \begin{subfigure}[t]{0.4\textwidth}
        \centering
        \includegraphics[width=\textwidth]{TMethod_kappa_loss_Vs_EBin_Canv_EG-11p2}
        \caption{}
    \end{subfigure}
    \hspace{1cm}
    \begin{subfigure}[t]{0.4\textwidth}
        \centering
        \includegraphics[width=\textwidth]{TMethod_kappa_loss_Vs_EBin_Canv_EG-6p0}
        \caption{}
    \end{subfigure}
\caption[]{}
\label{fig:EBinKloss}
\end{figure}
\end{landscape}





\begin{table}[h]
\centering
% \setlength\tabcolsep{10pt}
\renewcommand{\arraystretch}{1.2}
\begin{tabularx}{0.9\linewidth}{XHGHG}
  \hline
    \multicolumn{5}{c}{\textbf{Residual Gain Tests}} \\
  \hline\hline
            & \multicolumn{2}{c}{$\chi^{2}_{min}$} & \multicolumn{2}{c}{flat $\kappa_{loss}$} \\
  \cmidrule(lr){2-3}\cmidrule(lr){4-5}
    Dataset & \multicolumn{1}{c}{$A_{g} \times 10^{-4}$} & \multicolumn{1}{c}{p value} & \multicolumn{1}{c}{$A_{g} \times 10^{-4}$} & \multicolumn{1}{c}{p value} \\
  \hline
    60h & 8.5 & 0.81 & 8.5 & 0.81 \\
    HighKick & 5.1 & \sim 0 & 7.5 & 0.11 \\
    9d & 8.4 & 0.17 & 10.0 & 0.25 \\
    Endgame & 11.2 & \sim 0 & 6.0 & 0.04 \\
  \hline
\end{tabularx}
\caption[]{}
\label{tab:AdHocGainTests}
\end{table}


\begin{table}[h]
\centering
% \setlength\tabcolsep{10pt}
\renewcommand{\arraystretch}{1.2}
\begin{tabularx}{0.9\linewidth}{XOOJ}
  \hline
    \multicolumn{4}{c}{\textbf{Systematic Error Due to Residual Gain}} \\
  \hline\hline
    Dataset & \thead{Fit Method} & \multicolumn{1}{c}{$ \chi^{2}_{min} \quad \delta R$} & \multicolumn{1}{c}{flat $\kappa_{loss}$ $\boldsymbol{\delta R}$}  \\
  \hline
    \multirow{2}{*}{60h} & T & 28.7 & 28.7 \\
                         & R & 22.9 & 22.9 \\
  \hline
    \multirow{2}{*}{HighKick} & T & 1.0 & 11.8 \\
                              & R & 1.8 & 15.3 \\
  \hline
    \multirow{2}{*}{9d} & T & 18.1 & 24.4 \\
                        & R & 15.2 & 19.1 \\
  \hline
    \multirow{2}{*}{Endgame} & T & 29.9 & 18.8 \\
                             & R & 8.5 & 6.7 \\
  \hline
\end{tabularx}
\caption[]{}
\label{tab:AdHocGainErr}
\end{table}



