%!TEX root = ../BUSystematics.tex

\graphicspath{{Body/Figures/Ratio/}}

\section{Miscellaneous}


- give a short intro to this section


\subsection{Ratio Construction Systematic Uncertainties}

In the Ratio Method the \gmtwo period and the muon lifetime are taken to be known a priori. If the parameters are incorrectly chosen then it is possible there will be a systematic shift on \R. See Section 5.5.5 in \refref{phdthesis:2020Kinnaird} for a full description of the techniques used to estimate the sensitivities of \R to these two parameters, and the expected uncertainties in the parameters. The uncertainties on the period and lifetime are \SI{21.7}{ppm} and \SI{<10}{ns} respectively, and the sensitivities in the various datasets are given in \tabref{tab:ratioConstructionParsScan}. Multiplying the sensitivities to the \gmtwo period by the uncertainty, the systematic Uncertainties are 0.7, 2.3, 0.9, and \ppb{1.0} for the 60h, HighKick, 9d, and Endgame datasets respectively. Since there is no reason the datasets should treat this parameter differently, the largest number at 2.3 ppb is taken as the systematic uncertainty for all datasets. Regarding the systematic uncertainty for the choice of muon lifetime, the slopes are so small and the uncertainty on the lifetime is so small such that the systematic uncertainty is completely negligible, and is taken to be 0 for all datasets.



\begin{table}
\centering
\setlength\tabcolsep{20pt}
\renewcommand{\arraystretch}{1.2}
\begin{tabular*}{0.7\linewidth}{@{\extracolsep{\fill}}lcHH}
  \hline
    \multicolumn{4}{c}{\textbf{Sensitivity to Ratio Construction Parameters}} \\
  \hline\hline
    Dataset & & \multicolumn{1}{c}{$dR/d_{T_{a}}$} & \multicolumn{1}{c}{$dR/d_{\tau_{\mu}}$} \\
  \hline
    60h & & 0.034 & -1.336 \\
    HighKick & & -0.105 & -5.914 \\
    9d & & 0.042 & 0.546 \\ 
    Endgame & & 0.044 & 1.705 \\
  \hline
\end{tabular*}
\caption[Sensitivities of $R$ to ratio construction parameters]{Sensitivities of $R$ to ratio construction parameters. $dR/d_{T_{a}}$ is in units of ppb/ppm, while $dR/d_{\tau_{\mu}}$ is in units of \SI{}{ppb/ \micro s}. In both cases the sensitivities are extremely small.}
\label{tab:ratioConstructionParsScan}
\end{table}






\subsection{Clock and Timing Uncertainties}

\begin{table}
\centering
\setlength\tabcolsep{10pt}
\renewcommand{\arraystretch}{1.2}
\begin{tabular*}{1\linewidth}{@{\extracolsep{\fill}}lH}
  \hline
    \multicolumn{2}{c}{\textbf{Clock and Timing Uncertainties}} \\
  \hline
    Quantity & \thead{Value} \\
  \hline
    Input Clock Stability & 0.1 \\
    Input Clock Upconversion Factor & 2.1 \\
    Cluster Time Assignment & 1.0 \\
  \hline 
\end{tabular*}
\caption[]{clock errors - same for all datasets and methods - David determined clock errors \cite{phdthesis:2020Sweigart}}
\label{tab:clockErrs}
\end{table}


- not sure what to cite for the 1 ppb cluster time assignment error - pretty sure it's ``bounded at 1 ppb ...''



\subsection{Randomization}


-what about randomization? probably should talk about that some - not really a systematic error but it is if we want to combine results with someone else

- 200 random seeds

\begin{table}
\centering
\renewcommand{\arraystretch}{1.2}
\begin{tabularx}{\linewidth}{@{\extracolsep{\fill}}XYY}
  \hline
    \multicolumn{3}{c}{\textbf{Systematic Error due to Randomization}} \\
  \hline\hline
    Dataset & \thead{T-Method} & \thead{R-Method} \\
  \hline
    60h & 20.2 & 22.5 \\
    HighKick & 19.1 & 20.1 \\
    9d & 16.0 & 18.1 \\ 
    Endgame & 10.9 & 12.1 \\
    Endgame at \mus{50} & 12.3 & 13.7 \\
  \hline
\end{tabularx}
\caption[]{Units are in ppb.}
\label{tab:systematicError_Rand}
\end{table}
