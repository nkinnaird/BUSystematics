%!TEX root = ../BUSystematics.tex

\graphicspath{}

\section{Introduction}


This document describes the systematic uncertainty evaluations for the BU Run~1 \wa analysis effort. Run~1 consisted of four datasets; these datasets are called the 60h, HighKick (HK), 9d, and Endgame (EG), or 1a through 1d as is used in more recent documentation (and the upcoming publications). The former are used in this document. T- and R- (ratio) methods were used to fit the data. In all cases the systematic uncertainties are significantly smaller than the statistical uncertainties. Four main categories of systematic uncertainties include uncertainties due to gain corrections, the pileup correction, fitting for the coherent betatron oscillation (CBO), and muon loss cuts. Also included are some miscellaneous errors related to the construction of the ratio in the R-Method, clock and timing parameters, and randomization of cluster times. Systematic uncertainties due to corrections and effects outside the \wa analysis are not described or included in this document. These include the phase-acceptance effect, muon loss phase effect, E-field and pitch corrections and differential decay.

The author's thesis \cite{phdthesis:2020Kinnaird} includes previous estimates of most of the systematic uncertainties for the R-Method, as well as details on the analysis of the four datasets. The analysis effort has improved since the release of that thesis in January of 2020, including an analysis of the final HighKick dataset, results from T-Method fits to the data, and improvements to several systematic uncertainty evaluations. That document however is still one of the primary references for details about the BU \wa analysis, and no systematic uncertainties have changed significantly in magnitude since then.

One note of importance is that the collaboration decided on a later fit start time for the Endgame dataset, near \mus{50} as opposed to \mus{30}. This was done in order to reduce the phase-acceptance correction and associated error. This decision was made after the systematic uncertainties were estimated for the Endgame dataset, and so the uncertainties as calculated with the \mus{30} fit start time were applied to the results from the \mus{50} fit start time. This is justified as the systematic uncertainties for most sources are only expected to decrease as the time in-fill increases. Only a few which were re-calculated for those which it was unclear whether they would increase or decrease, or in the case of the randomization uncertainty where it was known that the error would increase.

The final systematic uncertainties are given in \tabref{tab:totalErrs}. These numbers can also be found in the final Run~1 systematic uncertainty spreadsheet, which also contains all systematic uncertainties for other Run~1 analysis groups \cite{UncertaintySpreadsheet}.



% - in the language for the whole document, might want to be a little more careful with my langauge, active voice, systematic ``unceratinty'' rather than ``error'', etc.

% - this document contains errors calculated within the analysis, and does not detail the errors like the phase acceptance, muon loss phase, etc. which are calculated outside the analysis (and some of these errors are larger - probably cite something here pointing to these guys or preliminary estimates of these guys)



