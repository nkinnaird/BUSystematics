%!TEX root = ../BUSystematics.tex

\graphicspath{}

\section{Introduction}


This document describes the systematic uncertainty evaluations for the BU Run~1 \wa analysis effort. Run~1 consisted of four datasets. These datasets are called the 60h, HighKick, 9d, and Endgame, or 1a through 1d as is used in more recent documentation (and the upcoming publications). The former will be used in this document. T- and R- (ratio) methods were used to fit the data. In all cases the systematic uncertainties are significantly smaller than the statistical uncertainties. Four main categories of systematic uncertainties include uncertainties due to gain corrections, the pileup correction, fitting for the coherent betatron oscillation (CBO), and muon loss cuts. Also included are some miscellaneous errors related to ratio construction, clock and timing paramters, and randomization of cluster times. Systematic uncertainties due to effects outside the \wa analysis are not described or included in this document, those including the phase-acceptance effect, muon loss phase effect, E-field and pitch corrections, differential decay, etc.

The author's thesis \cite{phdthesis:2020Kinnaird} includes previous estimates of most of the systematic uncertainties as well as details on the analysis of the four datasets. The analysis effort has improved since the release of that thesis in January of 2020, including an analysis of the final HighKick dataset, use of the T-Method to analyze the data, and improvements to several systematic uncertainty evaluations. That document however is still one of the primary references for details about the BU \wa analysis, and no systematic uncertainties have changed significantly in magnitude since then.



-need to define R somewhere, and probably cite it too - check statistal correlations note





% The four datasets for the \Rone analysis consisted of the 60h, HighKick, 9d, and Endgame datasets. These are sometimes abbreviated in different manners, such as 60h, HK, 9d, EG, or 1a, 1b, 1c, 1d respectively. The former is used at times in this document. The final, commonly blinded, best-fit \R values for all eleven analyses of all four datasets are given in \tabref{tab:analysisRValues}, where the equation
% \begin{align}
%   \omega_{a} = 2\pi \cdot \SI{0.2291}{MHz} \cdot (1 + (R - \Delta R) \times 10^{-6}),
% \label{eq:wa}
% \end{align}
% gives the relationship between \wa and \R, and $\Delta R$ is the common blinding offset.


% - in the language for the whole document, might want to be a little more careful with my langauge, active voice, systematic ``unceratinty'' rather than ``error'', etc.

% - this document contains errors calculated within the analysis, and does not detail the errors like the phase acceptance, muon loss phase, etc. which are calculated outside the analysis (and some of these errors are larger - probably cite something here pointing to these guys or preliminary estimates of these guys)

% - in all cases the systematic errors are much less than the statistical (cite statistical somewhere? probably David's combination note)


