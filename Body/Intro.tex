%!TEX root = ../BUSystematics.tex

\graphicspath{}

\section{Introduction}


This document describes the systematic uncertainty evaluations for the Boston University (BU) Run~1 \wa analysis. Run~1 consisted of four datasets; these datasets are called the 60h, HighKick (HK), 9d, and Endgame (EG), or 1a through 1d as is used in more recent documentation (and the upcoming publications)\footnote{The former are used in this document.}. T- and R- (ratio) methods were used to fit the data. The analysis was done on the ReconWest production clusters. Four main categories of systematic uncertainties on the \wa fits include uncertainties due to gain corrections, the pileup correction, fitting for the coherent betatron oscillation (CBO), and muon loss cuts. Also included are some miscellaneous uncertainties related to the construction of the ratio in the R-Method, clock and timing parameters, and randomization of cluster times. Systematic uncertainties due to corrections and effects outside the \wa analysis are not described or included in this document. These include the phase-acceptance effect, muon loss phase-momentum correlation effect, E-field and pitch corrections, and differential decay (and possibly others).

The author's dissertation \cite{phdthesis:2020Kinnaird} includes previous estimates of most of the systematic uncertainties for the R-Method, as well as extensive details on the BU analysis of the four datasets. The analysis has improved since the release of that dissertation in January of 2020, including an analysis of the final HighKick dataset, results from T-Method fits to the data, and improvements to several systematic uncertainty evaluations. That document however is still one of the primary references for details about the BU \wa analysis, and no systematic uncertainties have changed significantly in magnitude since it's release.

One note of importance is that the collaboration decided to use a later fit start time for the Endgame dataset, near \mus{50} as opposed to \mus{30}. This was done in order to reduce the phase-acceptance correction and associated uncertainty. This decision was made after the systematic uncertainties were estimated for the Endgame dataset using a \mus{30} fit start time, and so those uncertainties were applied to the results from fits starting near \mus{50}. This was justified as the systematic uncertainties for most sources were only expected to decrease as the time in-fill increases. Only a few uncertainties were re-calculated where it was unclear whether they would increase or decrease, or in the case of the randomization uncertainty, where it was known that the uncertainty would increase.

The final systematic uncertainties for the four datasets and two fit types are given in \tabref{tab:totalErrs} at the end of this document. These numbers can also be found in the final Run~1 systematic uncertainty spreadsheet, which also contains all systematic uncertainties for other Run~1 analysis groups \cite{UncertaintySpreadsheet}. In all cases the systematic uncertainties were found to be significantly smaller than the statistical uncertainties. For more details on the Run~1 analysis in general, and on analyses performed by other groups, there is a very useful readers guide available compiled by T. Gorringe \cite[and references therein]{ReadersGuide}.

