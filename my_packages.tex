\usepackage[intlimits]{amsmath}
\usepackage{mathtools} % for splitfrac
\usepackage[utf8]{inputenc}
\DeclareUnicodeCharacter{2212}{-}
\usepackage{amsfonts,amssymb}
\DeclareSymbolFontAlphabet{\mathbb}{AMSb}
\usepackage{slashed}
\usepackage{textcomp}

\usepackage{float}
\usepackage[]{caption,subcaption}
\setcaptionmargin{0.5in}

\usepackage{fancyhdr}
%\usepackage{fancyheadings} % what does this do?
\usepackage{fancybox}

\usepackage{ifthen}
\usepackage{lscape,afterpage}
\usepackage{xspace}

\usepackage{siunitx}
\sisetup{detect-weight=true, detect-family=true}

\usepackage{xcolor}

\usepackage{appendix}

\usepackage{enumitem}

% table packages and commands
\usepackage{tabularx,booktabs}
\usepackage{dcolumn}
\usepackage{multirow}

\usepackage{longtable}

\usepackage{pdflscape}
\usepackage{afterpage} % can maybe be used to deal with text around landscape tables if I have troubles down the line
\usepackage{rotating}

% https://tex.stackexchange.com/questions/13509/biblatex-in-a-nutshell-for-beginners
% if I get the error, ".bbl not created by bib latex", run biber $, where $ is the basename of the main .tex file, then rerun the build
\usepackage[sorting=none]{biblatex}
\addbibresource{myBib.bib}
% \addbibresource(othersfiles.bib) % if I ever split up the bibiliography add multiple resources like this

%feynman digram package
% \usepackage{tikz-feynman} % doesn't work out of the gate, wants LuaLaTex - see https://jpellis.me/projects/tikz-feynman/
% \tikzfeynmanset{compat=1.1.0} 
\usepackage[compat=1.1.0]{tikz-feynhand} % simpler version of tikz-feynman - see https://ctan.org/pkg/tikz-feynhand?lang=en

%==========================================================================%
%%% graphicx and pdf creation
\usepackage{graphicx}

\usepackage{hyperref} % set up this package last (or almost last anyways)
% \hypersetup{
%     colorlinks,
%     linkcolor={red!50!black},
%     citecolor={blue!50!black},
%     urlcolor={blue!80!black}
% }

% \usepackage{glossaries} % might be able to use this to make an alphabetically ordered list of abbreviations - needs to be set up after hyperref

\interfootnotelinepenalty=10000 % hack to force footnotes to one page

%==========================================================================%
%               USEFUL MACROS AND COMMANDS
%==========================================================================%

\newcommand{\figref}[1]{Figure~\ref{#1}}
\newcommand{\chapref}[1]{Chapter~\ref{#1}}
\newcommand{\secref}[1]{Section~\ref{#1}}
\newcommand{\appref}[1]{Appendix~\ref{#1}}
\newcommand{\refref}[1]{Reference~\cite{#1}}
\newcommand{\equref}[1]{Equation~\ref{#1}}
\newcommand{\tabref}[1]{Table~\ref{#1}}

\newcommand{\latex}{\LaTeX\xspace}
\newcommand{\ROOT}{\texttt{ROOT}\xspace}


\def\BU{BOSTON UNIVERSITY}
\def\Bu{Boston University}
\def\GSA{GRADUATE SCHOOL OF ARTS AND SCIENCES}
\def\Gsa{Graduate School of Arts and Sciences}

\def\wa{$\omega_{a}$\xspace}
\def\chisq{$\chi^{2}$\xspace}
\def\gmtwo{$g-2$\xspace}
\def\Ta{$T_{a}$\xspace}
\def\Tatwo{$T_{a}/2$\xspace}
\def\amu{$a_{\mu}$\xspace}
\def\g{$g$\xspace}

\def\mutau{\SI{2.2}{\micro s}\xspace}
\def\taumu{$\tau_{\mu}$\xspace}

\newcommand{\ns}[1]{\SI{#1}{ns}\xspace}
\newcommand{\mus}[1]{\SI{#1}{\micro s}\xspace}
\newcommand{\mum}[1]{\SI{#1}{\micro m}\xspace}

\def\eV{\text{e\kern-0.15ex V}\xspace}
\def\keV{\text{k\eV}\xspace}
\def\MeV{\text{M\eV}\xspace}
\def\GeV{\text{G\eV}\xspace}
\def\TeV{\text{T\kern-0.1ex \eV}\xspace}

\def\DT{$\Delta t_{12}$\xspace}

\def\R{$R$\xspace}
\def\K{$\kappa_{loss}$\xspace}

\newcolumntype{L}{D{.}{.}{2,3}} % define a column type L with specific spacing before and after the decimal point
\makeatletter
\newcolumntype{B}{>{\boldmath\DC@{.}{.}{2,3}}l<{\DC@end}} % define a column that's bold and does the above
\makeatother

\newcolumntype{F}{D{.}{.}{3,1}}
\newcolumntype{E}{D{.}{.}{3,3}}

\newcolumntype{H}{D{.}{.}{1,1}}
\makeatletter
\newcolumntype{J}{>{\boldmath\DC@{.}{.}{1,1}}l<{\DC@end}} % define a column that's bold and does the above
\makeatother

\newcolumntype{G}{D{.}{.}{2,1}}
\makeatletter
\newcolumntype{K}{>{\boldmath\DC@{.}{.}{2,1}}c<{\DC@end}} % define a column that's bold and does the above
\makeatother

% \newcolumntype{N}{D{.}{.}{2,1}}
% \newcolumntype{Y}{>{\centering\arraybackslash}X}

% \makeatletter
% \newcolumntype{Y}{>{\DC@{.}{.}{2,1}}X<{\DC@end}}
% \makeatother

\newcolumntype{Y}{D..{2.1}}

% \newcolumntype{Y}{>{\centering\arraybackslash}D{.}{.}{2,1}}


\usepackage{array}
% \newcolumntype{R}[1]{>{\centering\arraybackslash}p{#1}} % for wrapping text
\newcolumntype{R}[1]{>{\raggedright\let\newline\\\arraybackslash\hspace{0pt}}m{#1}}

% command to make row in table bold
\newcommand\setrow[1]{\gdef\rowmac{#1}#1\ignorespaces}
\newcommand\clearrow{\global\let\rowmac\relax}

\newcommand*{\thead}[1]{\multicolumn{1}{c}{#1}} % define a new command thead which centers table headers in the first row and works with the rest of the table environment

